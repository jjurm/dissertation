\section{Preparation / libraries / frameworks}
\input{content/3implementation-preparation}

\section{Data model}

\subsection{Entities}
\begin{enumerate}
    \item GeneratedGraph
    \item MetricExperiment
\end{enumerate}

\subsection{Database}

\subsubsection{Storing graphs in database}

For evaluation of robustness, we need to be able to compare either values or ranks of values of a specific node between two generated graphs, therefore we need to be able to preserve mapping of nodes of a generated graph to nodes in the original dataset.
Thus, a node ID originating from the original dataset must be stored, not just values of a metric for each node.

I considered the following formats of storing graph:
\begin{enumerate}
    \item DGS
    \item DOT - doesn't preserve node IDs
\end{enumerate}


\section{Loading graphs}


\section{Generating graphs}

For evaluating robustness of graph metrics according to the model described in the~\nameref{ch:proposal}, we need to evaluate graph metrics on a number of similar graphs - graphs that all describe the same facts from the real world.
We need to be able compare value of a metric between different graphs that \textit{share the same source} or are \textit{of the same foundation}.

One specific example may be a graph constructed from social network, such as Facebook.
Let $F_0$ be a graph constructed from people and their mutual friendships at Facebook, and let $G_1$ be a graph constructed from people, with the set of edges including stronger friendships. $G_0$ and $G_1$ have the same nodes, but edges of $G_1$ is a subset of edges of $G_0$.
Now, $G_0, G_1$ are different but describe the same structure of people in the world, i.e.\ convey the same meaning.

\textbf{Preserving node identities} For two such graphs, we can define subsets $V_{c0}$ and $V_{c1}$ of nodes of the respective graphs, such that there exists a bijection $V_{c0} \leftrightarrow V_{c1}$.
These nodes will have important meaning for definition of the robustness function, because the pairs of corresponding nodes describe the same entities of the real world.
Thus, we can observe how a graph metric behaves for a particular node in different graphs.

\begin{description}
    \item[Conjugate graphs] Define graphs $G_0, G_1$ to be \textbf{conjugate} if they are generated from the same source graph and have a common subset of nodes and edges.
\end{description}

The goal of this section is to present possible algorithm(s) for taking a source graph and generating

\subsection{Random edge deletion}

\subsection{Thresholding edges of protein graphs}


\section{Evaluating metrics}


\section{Metric robustness}


\section{Visualising graphs}


\section{Command line interface}

