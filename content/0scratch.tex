\chapter{(Scratch)}

\section{Libraries / frameworks}
I used a number of libraries to help with concepts as described in the subsections below.

\subsection{Graph representation and algorithms}

I reviewed the most developed and maintained libraries that allow representing graphs.

\begin{itemize}
    \item \textbf{JGraphT} (\url{https://jgrapht.org})

    \enquote{a Java library of graph theory data structures and algorithms}

    Probably the most popular Java library for working with graphs.
    It is algorithm-focused, however, the algorithms relate mostly to walking on graph nodes and are not for evaluating graph metrics.
    JGraphT also does doesn't come with visualisations.

    \item \textbf{GraphStream} (\url{http://graphstream-project.org/})



    \item \textbf{Graphviz} (\url{https://www.graphviz.org/})

    \enquote{open source graph visualization software}

    Graphviz is specifically made for creating visual graphs, diagrams and abstract networks, for visual interfaces.
    Not suitable for this project.
\end{itemize}

Taking pros and cons of the above-mentioned graph libraries into account, I chose to use \textbf{GraphStream}.

GraphStream consists of 3 modules:

\begin{enumerate}
    \item \texttt{gs-core} Core implementation of the graph library
    \item \texttt{gs-algo} Various algorithms from graph theory
    \item \texttt{gs-ui} Components for visualising graphs
\end{enumerate}

\subsection{Batch Job processing framework}

As evaluation of metrics on a huge number of generated graphs takes a lot of computational resources, \texttt{graffs} runs on a framework that supports distributed computations.

\begin{itemize}
    \item Apache Flink (\url{https://flink.apache.org})

    \enquote{Stateful Computations over Data Streams}

    Slightly less popular framework for distributed streaming and data processing.
    \item Apache Samza (\url{http://samza.apache.org/})

    \enquote{open-source near-realtime, asynchronous computational framework for stream processing}

    The usage seems to be relatively heavyweight and relies on Apache Kafka and Zookeeper.
    \item Apache Storm (\url{https://storm.apache.org/}) *

    \enquote{open source distributed realtime computation system}

    (a good possible alternative.
    See \url{https://storm.apache.org/releases/2.1.0/Tutorial.html})
    \item Apache Akka (\url{https://akka.io/})

    \enquote{toolkit and runtime simplifying the construction of concurrent and distributed applications on the JVM}
    \item Apache Hadoop (\url{https://hadoop.apache.org/})

    \enquote{framework that allows for the distributed processing of large data sets across clusters}
    \item Apache Heron (\url{https://apache.github.io/incubator-heron/})

    \enquote{realtime, distributed, fault-tolerant stream processing engine from Twitter}
    \item Apache Beam (\url{https://beam.apache.org/})

    \enquote{open source unified programming model to define and execute data processing pipelines}
    \item Apache NiFi (\url{https://nifi.apache.org/})

    \enquote{supports powerful and scalable directed graphs of data routing, transformation, and system mediation logic}
    \item Spring Boot (\url{https://spring.io/projects/spring-boot})

    \enquote{makes it easy to create stand-alone, production-grade Spring based Applications}
    \item Apache Apex (\url{https://apex.apache.org/})

    \enquote{Enterprise-grade unified stream and batch processing engine} (now a retired project)
    \item Apache Spark™

    \enquote{unified analytics engine for large-scale data processing}

    Spark is one of the most maintained framework for computation-heavy tasks
\end{itemize}

\subsection{Database}

\subsubsection{Database backend}

For simplicity, the H2 database system will be used, which can store a database in a single file.
A good alternative would be a MySQL installation.

\subsubsection{Object-relational mapping (ORM) library}

For robustness and simplicity of usage, \texttt{graffs} uses Java Persistence API (JPA) to help persist Java objects in the database as well as query them.
The API is Java classes and methods, so it abstracts away any queries in the native language of the database (such as SQL).

The \texttt{Hibernate} library (\url{http://hibernate.org/}) is an implementation of JPA that I used for the project.
It hides the underlying data storage system, as well as creating schema for the underlying database.
Schema is derived from Java classes (entities) defined in the source code.

Hibernate can also seamlessly connect to the H2 backend.

\subsection{CLI processing}

The \texttt{Clikt} library helps write the command line interface classes in a structured way.
Clikt is written in Kotlin and works especially well with Kotlin code as it leverages many of its features.
Further details of how I developed the command-line interface is in the~\nameref{ch:implementation} chapter.

\subsection{Apache Libraries}

\begin{itemize}
    \item Apache Commons Configuration:

    \url{https://commons.apache.org/proper/commons-configuration/}
    \item Apache Commons Compress:

    \url{https://commons.apache.org/proper/commons-compress/}
    \item Apache Commons CSV: \url{https://commons.apache.org/proper/commons-csv/}
    \item Apache Commons Logging:

    \url{https://commons.apache.org/proper/commons-logging/}
    \item Apache Commons Math: \url{https://commons.apache.org/proper/commons-math/}
    \item Apache Commons Pool: \url{https://commons.apache.org/proper/commons-pool/}
    \item Apache Commons Statistics:

    \url{https://commons.apache.org/proper/commons-statistics/}
\end{itemize}


\section{Terms and definitions}

I will loosely clarify, concretise or define the following terms:

\begin{description}
    \item[Graph]
    By graph, I mean a mathematical model of nodes connected by edges, from graph theory.
    Graphs is this work are implicitly \textbf{undirected}, if not stated otherwise.
    Nodes and edges may have values.

    \item[Metric of a graph]
    or graph metric, or just metric (in some contexts: measure), is a function of graphs.
    A metric may return a single value, or multiple values, such as for each node or each edge.
    Values are not necessarily numerical.

    \item [Dataset]
    In the context of this work, a dataset is a graph (alternatively a collection of graphs) that has a semantic meaning.
    For example, it may be known what source the graph came from, such as a social network.

    \item[Perturbation of a graph]
    Any change to the input graph resulting in a new graph (such as deleting a small percentage of edges), along with a mapping between nodes and edges of the original graph that have been preserved in the produced graph.

    \item[Generated graph]
    A graph that was produced by perturbing a dataset

    \item[Robustness of graph metric]
    A property of a graph metric, determining how sensitive to some perturbations in the input graph the metric is, i.e. how much the metric value(s) will change with small changes in the input graph.
    Robustness of a metric may depend on the kind of graph it is used on.
\end{description}

The~\nameref{ch:implementation} chapter further explain in detail terms \textbf{graph perturbations}, \textbf{metric value change} as well as \textbf{robustness}.
