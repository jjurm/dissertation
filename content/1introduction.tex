\epigraph{A reported value whose accuracy is entirely unknown\\ is worthless.}{Churchill Eisenhart~\cite{EisenhartExpressionUncertaintiesFinal1968}}

%https://books.google.sk/books?id=8qIZuhFlAK0C&pg=PA1067&lpg=PA1067&source=bl&ots=xBMuLQiWbg&sig=ACfU3U2FRfryqQk1Cy5ZFSSK32aEHEhFnQ&hl=en&sa=X&ved=2ahUKEwj0yq7mobnpAhUCQhUIHSzWBNEQ6AEwDHoECAkQAQ#v=onepage&q=churchill%20eisenhart%20quotes&f=false
%
%Snedecor, G. W. Statistician
%The characteristic which distinguishes the present-day professional statistician, is his interest and skill in the measurement of the fallibility of conclusions.
%On a Unique Feature of Statistics Presidential Address to the American Statistical Association, December 1948 Journal of the American Statistical Association , Volume 44, Number 245, March 1949
%Gaither, Carl C., and Alma E. Cavazos-Gaither. The Gaither's Dictionary of Scientific Quotations : A Collection of Quotations Pertaining to Archaeology, Architecture, Astronomy, Biology, Botany, Chemistry, Cosmology, Darwinism, Death, Engineering, Geology, Life, Mathematics, Medicine, Nature, ..., Springer, 2008. ProQuest Ebook Central, http://ebookcentral.proquest.com/lib/cam/detail.action?docID=603589.
%p1525
%
%Hooke, Robert English physicist 1635– 1703
%It is commonly believed that anyone who tabulates numbers is a statistician. This is like believing that anyone who owns a scalpel is a surgeon.
%How to Tell the Liars from the Statisticians Chapter 1 (p. l) Marcel-Dekker, Inc. New York, New York, USA. 1983
%Gaither, Carl C., and Alma E. Cavazos-Gaither. The Gaither's Dictionary of Scientific Quotations : A Collection of Quotations Pertaining to Archaeology, Architecture, Astronomy, Biology, Botany, Chemistry, Cosmology, Darwinism, Death, Engineering, Geology, Life, Mathematics, Medicine, Nature, ..., Springer, 2008. ProQuest Ebook Central, http://ebookcentral.proquest.com/lib/cam/detail.action?docID=603589.
%p1524


I designed a way and built a tool to analyse robustness of graph metrics by evaluating them on many generated graphs.

Graph metrics are often used to find and derive facts about important nodes or components of networks.
However, real-world datasets may be inherently imprecise, or the procedure of obtaining networks from raw measured data may introduce (sometimes invisible) inaccuracies.
When evaluating graph metrics on imprecise data, it is crucial to be able to reason about the reliability and of the results, as with any other statistical procedures.
The field of examining stability and reliability of graph metrics is relatively novel.

This project was inspired by, and is based on the paper ``Measuring rank robustness in scored protein interaction networks'' by L.~V.~Bozhilova et~al.~\cite{Bozhilova2019} (further referred to as ``The Paper'') which served as a kickoff point for ideas in this project.
The Paper introduces ways to assess \textsl{robustness} of graph metrics specifically on protein interaction networks with confidence-scored edges.

In this dissertation I summarise the research done on this topic and build a command-line tool \graffs that helps generalise the research of graph metric robustness and enables similar experiments on graphs of other kinds.
The program is written in Kotlin from scratch, and wrapped up and published as an open-source library\footnote{Available at: \url{https://github.com/jjurm/graffs}} that can further be used by future projects in this research area.

The goals of my work are, namely:
\begin{enumerate}[noitemsep,topsep=5pt]
    \item To implement a tool called \graffs to automate the kind of experiments done in The Paper
    \item To reproduce a subset of results from The Paper
    \item To extend the idea of graph metric robustness on unscored networks
    \item (Extension) to polish the tool and publish it under an open-source licence
\end{enumerate}

\Cref{ch:preparation} explains The Paper in more detail, revises the background material and explains method used in \graffs.
First, I generalise graph acquisition using edge score thresholding and random edge removal.
\Cref{ch:implementation} describes the tool implementation of the framework, which I then evaluate in \cref{ch:evaluation}, by applying a number of robustness measures on some of the most common node-level graph metrics.
Finally, I compare results with The Paper and derive new observations about unscored graph classes.
